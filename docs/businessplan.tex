\documentclass[11pt]{article}
%Gummi|065|=)



\usepackage{graphicx}




\begin{document}



\begin{titlepage}
    \begin{center}
        \vspace*{1cm}
        
        {\huge\textbf{Business Plan}
        
        \vspace{0.5cm}
        { \Large Applications in Practical High End Computing }
        
        \vspace{1.5cm}
        
        \textbf{Group 1}}
        
        \vfill
        
        
        \vspace{0.8cm}
        
        \includegraphics[width=0.4\textwidth]{images/cranfield}
        
        Software Techniques in Engineering \\
        MSc in Computational \& Software Techniques in Engineering \\
        Cranfield University\\
        March 2017
        
    \end{center}
\end{titlepage}


\section*{Executive Summary}

The task at hand is to improve a given program suit consisting of the programs SurfIT, MoveIt and FlexIT.
There are many ways of making the software better: Enhance code readability, fix minor bugs, improve the user interface but most importantly increase the performance. To gain performance we try to optimize the CUDA implementation but we also want to try the software on different systems to see how it influences the runtime. One of the systems we want to try is the Amazon Web Services (AWS) Cloud. They provide instances with dedicated GPUs on which we want to run FlexIT on. 


\section{Background and Purpose}
Designing and building an airplane is a highly expensive and time consuming process. First a model has to be built and put in a wind tunnel where its aerodynamics can be measured. This is a error prone and lengthly process. It is a lot cheaper to create a digital model and simulate the aerodynamics on a computer.

One such software to simulate aerodynamics is developed at Cranfield University by Dr Dominique Fleischmann. In the course of the Group Project several student teams will try their best to improve this software.




\section{Group Structure and Management}

Group 1 consists of 5 team members. Each of them has a specific role. There is one person working general code structure and readability, two people focus on the CUDA implementation, the mathematician of the group is working on the algorithm to see if it can be made more efficient and one member is working on software deployment on different systems. We all try to fix bugs in the process of working with the software. A bug tracking system was implemented.

All of us are committed to work together. If one team member encounters a problem he cannot solve then we will shift resources to tackle this problem together. 



\section{Market Analysis}

There are several cloud providers in the market. The three biggest ones are Amazon with their AWS, Google with their Google Cloud Platform and Microsoft with Azure.

For our project we need to access the visualized system on a low level since we need to install CUDA and QT libraries. Thus we require infrastructure as a service (IaaS). Google is focusing more on platform as a service (PaaS) and software as a service (SaaS). Microsoft offers hybrid solutions. 

We decided to use AWS for our project. Even though we require Windows to run the application Microsoft Azure is not the first choice. AWS also provides systems that run a Windows operating system. Furthermore we have worked with AWS in a previous course before. Therefore accounts already exist and knowledge about how to use the platform is present.


\section{Financial Forecast}

The host machine must have access to a GPU since the FlexIT program does calculations on GPUs. There are several different instance types available at the AWS. The one which will be used for this project is called \texttt{g2.2xlarge} and it is the cheapest instance with a dedicated GPU. The hourly cost for on demand operation of this instance is $0.767$ US Dollars. 

The expected up-time of the instance is estimated to be around 20 hours. Withing that time it should be possible to set up all the dependencies and either figure out that it is not possible to run FlexIT on this system or run a benchmark to analyse how the performance compares to a reference system (ie. the lab computer).

The cost on AWS EC2 would therefore amount to approximately 16\$. At the current exchange rate this would be about 13\pounds. There might be small additional charges for snapshot storage. However the use of expensive storage like RDS or EFS is not planned. 

In total the overall cost on AWS should not exceed 20\$ in the course of this project. Since this is a student project those costs should be covered by the university. 


\newpage
\section{Conclusion}


FlexIT is a tool for simulating aerodynamics. In order to figure out if it can be run in the cloud and how its performance would be influenced when doing so, Group 1 would like to deploy the program on AWS.
This requires the use of instances with Windows OS and a dedicated GPU. The expected cost to the university should not exceed 20 US dollars. 







\end{document}
